\documentclass{article}
\usepackage{graphicx}
\usepackage{float}
\usepackage{indentfirst}
\usepackage{chngpage}
\setlength{\parindent}{2em}
\usepackage[table,xcdraw]{xcolor}
\usepackage{geometry}
\usepackage{multirow}
\usepackage{authblk}
\usepackage[T1]{fontenc}
\usepackage[utf8]{inputenc}
%\usepackage{ctex}
\usepackage{color}
\usepackage{enumerate}
\usepackage{graphics}
\geometry{left=3.5cm,right=3.5cm,top=3.5cm,bottom=3.5cm}
%\usepackage[pdftex]

%\subsection{}
%\subsubsection{•}
%
%
%\indent 首行缩进并且在“\\”另起一行\\
%
%
%$$F = F_0(sin\omegat + \delta)$$%数学模式
%
%
%\begin{equation}
%I\frac{d^2\theta}{dt^2}=-k\theta-b\frac{d\theta}{dt}+\tau_0cos\omega t
%\end{equation}%这种可以自动编号
%
%
%%插入jpg图片
%\begin{figure}[H]%这个H一定要有
%\centering
%\includegraphics[width=2in,height=1.75in]{1.jpg}
%%\label{Figure.1}
%\end{figure}
%
%\begin{center}%这个可以把一部分东西居中
%Figure.1\\
%\end{center}
%
%%表格用https://www.tablesgenerator.com/
\title{Physics Laboratory\\[2ex]
\begin{center}
VP160 Project\\
Numerical solution of Newton's equations of motion\\
Group 26
\end{center}
}

\author{Liu Zhendong\\
Yuan Yin 517370910260\\
Zhang Xiuqi\\Zhou Zikun}

\begin{document} \maketitle
\begin{center}

We state that each of us has contributed equally to this project.\\
Signature:
\end{center}
\newpage % START THE DOCUMENT!

\section{Introduction}
\section{Problems}
\subsection{Projectile motion with air drag}
\subsubsection{Projectile motion being solved}
From paragraph 1-4 we consider the problem of a 2D projectile of mass m = 1 kg moving with a linear drag, moving close to the Earth’s surface. The equation of motion in this problem is\\
$$
\ddot{r}=-g-kv
$$
where κ = k/m and k is the drag coefficient. We assume that the projectile starts out at the origin with velocity $v_0 = 90 m/s$ at an angle $\alpha$ to the horizontal.\\
\paragraph{(1) Recursive Equations For Velocity and Position of Projectile Given By the Euler Method}
\paragraph{(2) Examination whether the numerical result is sensitive to the choice of the step $\Delta t$}
\paragraph{(3) Choosing appropriate value of the step and solve the problem numerically for five different $\alpha$.}
\paragraph{(4) Solving the problem numerically for five different drag coefficient k.}
\subsubsection{Projectile motion being solved}
From paragraph 5-9, we suppose that the same particle is subject to a quadratic drag. That is, its equation of motion reads\\
$$
\ddot{r} = -g- \beta |v| v
$$
where $\beta = b/m$ and b is the drag coefficient. We assume again that the projectile starts out at the origin with velocity $v_0 = 90 m/s$ at an angle $\alpha$ to the horizontal.\\
\paragraph{(5) Recursive equations for velocity and position of the projectile.}
\paragraph{(6) Examination whether the numerical result is sensitive to the choice of the step $\Delta t$ for fixed initial conditions.}
\paragraph{(7) Choosing appropriate value of the step and solve the problem numerically for five different $\alpha$.}
\paragraph{(8) Solving the problem numerically for five different drag coefficient k.}
\paragraph{(9) The trajectories of a projectile with different drag conditions}
\subsection{Simple harmonic oscillator}
\subsubsection{Projectile motion being solved}
In this part we numerically solve the equation of motion of the 1D simple harmonic
oscillator $\ddot{x} = −{\omega_0}^2x$, with the initial conditions $x(0) = 0.5 m$ and $v_x(0) = 1 m/s$. Assume the natural angular frequency $\omega_0 = 1.5 s−1$.\\
\paragraph{(1) Recursive equations for velocity and position of the projectile.}
\paragraph{(2) Examination whether the numerical result is sensitive to the choice of the step $\Delta t$ for fixed initial conditions.}
\paragraph{(3) Calculation and graph of E(t)}
\paragraph{(4) Recursive second-order Runge-Kutta equations for vi and xi}
\paragraph{(5) Condition if choosing the worst–performing step size $\Delta t$ }
\section{Accuracy of Numerical Methods}

\subsection{Factors Limiting the Accuracy of Numerical Methods}

\subsection{Order of Runge-Kutta method and Euler method}

\end{document}
